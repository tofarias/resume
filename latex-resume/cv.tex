%!TEX TS-program = xelatex
\documentclass[]{friggeri-cv}
\usepackage{afterpage}
\usepackage{hyperref}
\usepackage{color}
\usepackage{xcolor}
\hypersetup{
    pdftitle={},
    pdfauthor={},
    pdfsubject={},
    pdfkeywords={},
    colorlinks=false,       % no lik border color
   allbordercolors=white    % white border color for all
}
\addbibresource{bibliography.bib}
\RequirePackage{xcolor}
\definecolor{pblue}{HTML}{0395DE}

\begin{document}
\header{Tiago O.}{ de Farias}
      {Analista de Sistemas PHP}
      
% Fake text to add separator      
\fcolorbox{white}{gray}{\parbox{\dimexpr\textwidth-2\fboxsep-2\fboxrule}{%
.....
}}

% In the aside, each new line forces a line break
\begin{aside}
  \section{Endereço}  	
    Rua Doutor José Bento Corrêa, 545
    Bairro Protásio Alves,
    CEP: 91450-030
    Rio Grande do Sul, Porto Alegre
    ~
  \section{Telefone}
    51 9292 2705
    51 3018 1864
    ~
  \section{E-mail}
    \href{mailto:tiago.farias.poa@gmail.com}{\textbf{tiago.farias.poa@}\\gmail.com}
    ~
  \section{Web \& Git}
    \href{https://github.com/tofarias}{github.com/tofarias}
    \href{https://bitbucket.org/tiagofarias}{bitbucket.org/tiagofarias}
    \href{https://tiagoodefarias.wordpress.com/}{tiagoodefarias.wordpress}
    \href{https://www.facebook.com/groups/laravelrgs}{facebook.com/laravelrgs}
    ~
  \section{Programação}
    \includegraphics[scale=0.62]{img/techinical-skills.png}
    ~
  \section{Sistemas Operacionais}
    \textbf{GNU/Linux}\includegraphics[scale=0.40]{img/4stars.png}
    \textbf{Windows}\includegraphics[scale=0.40]{img/3stars.png}
    ~
  \section{Personal Skills}
    \includegraphics[scale=0.62]{img/personal.png}
    ~
\end{aside}

\section{Formação}
\begin{entrylist}
  \entry
    {2014 - 2016}
    {Pós-Graduação em Sist. de Informação com Métodos Ágeis}
    {UniRitter}
    {Tópicos principais: Práticas de Desenvolvimento Ágil, Design Patterns, Smells e Refatoring, Arquiteturas para Web, Testes Ágeis de Software, Gerência de Projetos, Agile Coaching e Mentoring.\\
    \emph{Título do Artigo: "Aplicacação de Domain-Driven Design no Gerenciamento de
GRU de Cronotacógrafo no Inmetro/RS".}\\
    \emph{Orientador: M.e. Guilherme Silva de Lacerda.}\\}
  \entry
    {2005 - 2009}
    {Análise e Desenvolvimento de Sistemas}
    {Università di Pisa, Italy}
    {Main subjects: Matematics and Physics, Programming, Operational Research, Telecommunication Systems, Digital and Analogical Electronics.\\
    \emph{Title of the Thesis: "Development, Management and Migrations of web contents and applications".}\\
    \emph{Thesis activity carried out during an internship period at Atitlan Engineering SRL.}\\}
  \entry
    {2000 - 2005}
    {Scientific Disploma.}
    {Liceo Scientifico, Matera, Italy}
    {Scientific Secondary School.\\
    Main subjects: Matematics, Physics, Computer Science.}
\end{entrylist}

\section{Experience}
\begin{entrylist}
  \entry
    {02/13 - Now}
    {Software Engineer}
    {Fluidmesh Networks SRL, Milano, Italy}
    {Design and development of algorithms, solutions and high speed mobility           schemes for Wireless Mesh Networks. Design and development of embedded software     for Wireless Network devices.\\}
  \entry
    {01/12 - 01/13}
    {Freelance Developer \& Consultant}
    {Icosaedro Solutions}
    {Design and development of Android Applications, Web Solutions, Unix and GNU/Linux software.\\}
    \entry
    {12/09 - 06/09}
    {Project Manager and Webmaster}
    {D.I.D.A.G., Grassano (MT), Italy}
    {Design, development and management of an e-commerce website on Joomla!1.5 CMS platform.\\}
    \entry
    {06/09 - 09/09}
    {Part-time collaboration}
    {Area Sistemi Informatici, Università di Pisa, Italy}
    {Computer technical support. Problem solving related to hardware, software and Operating Systems. Management of the internal network.\\}
    \entry
    {06/09 - 09/09}
    {Internship}
    {Atitlan Engineering SRL, Pisa, Italy}
    {Management and migration of servers. Development of web templates and interfaces. Management of SQL databases.}
\end{entrylist}



\section{Certifications}
\begin{entrylist}
  \entry
    {02/2013}
    {Intro to Computer Science}
    {Udacity. E-learning}
    {\emph{Building a Python Search Engine}}
\end{entrylist}

\newpage

\begin{aside}
~
~
~
  \section{Places Lived}
    \includegraphics[scale=0.25]{img/italia.png}
    ~
  \section{Languages}
    \textbf{Italian}\includegraphics[scale=0.40]{img/5stars.png}
    \textbf{English}\includegraphics[scale=0.40]{img/4stars.png}
\end{aside}

\section{Publications}
C. Benedetto, E. Mingozzi, C. Vallati\\
\textbf{A Handoff Algorithm based on Link Quality Prediction for Mass Transit Wireless Mesh Networks}\\
\emph{Proceedings of the 18th IEEE Symposium on Computers and Communications (ISCC 2013), Split, Croatia, July 7-10, 2013}
\\
\section{Other Info}
For the Italian job market:\\
\emph{Si autorizza il trattamento delle informazioni contenute nel curriculum in conformità alle disposizioni previste dal d.lgs. 196/2003. Si dichiara altresì di essere consapevole che, in caso di dichiarazioni non veritiere, si è passibili di sanzioni penali ai sensi del DPR 445/00 oltre alla revoca dei benefici eventualmente percepiti.}
\\
\begin{flushleft}
\emph{January 14th, 2014}
\end{flushleft}
\begin{flushright}
\emph{Carmine Benedetto}
\end{flushright}

%%% This piece of code has been commented by Karol Kozioł due to biblatex errors. 
% 
%\printbibsection{article}{article in peer-reviewed journal}
%\begin{refsection}
%  \nocite{*}
%  \printbibliography[sorting=chronological, type=inproceedings, title={international peer-reviewed conferences/proceedings}, notkeyword={france}, heading=subbibliography]
%\end{refsection}
%\begin{refsection}
%  \nocite{*}
%  \printbibliography[sorting=chronological, type=inproceedings, title={local peer-reviewed conferences/proceedings}, keyword={france}, heading=subbibliography]
%\end{refsection}
%\printbibsection{misc}{other publications}
%\printbibsection{report}{research reports}

\end{document}
